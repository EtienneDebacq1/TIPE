\documentclass[aspectratio=43,xcolor=dvipsnames]{beamer}
\usetheme{SimpleDarkBlue}
\usepackage{xcolor}
\usepackage{subfig}
\usepackage{hyperref}
\usepackage{graphicx} 
\usepackage{booktabs}
\usepackage{tikz}

\captionsetup[subfigure]{font=small, labelformat=empty}

\usepackage{algorithm}
\usepackage{algpseudocode}
\usepackage{amsmath}
\usepackage{float}

\setbeamertemplate{footline}[default]

\setbeamertemplate{footline}{
  \leavevmode
  \hfill
  \usebeamercolor[fg]{page number in head/foot}
  \usebeamerfont{page number in head/foot}
  \insertframenumber
  \hspace{1em}
  \vspace{1em}
}

\usepackage{listings}

\definecolor{codegreen}{rgb}{0,0.6,0}
\definecolor{codegray}{rgb}{0.5,0.5,0.5}
\definecolor{codepurple}{rgb}{0.58,0,0.82}
\definecolor{backcolour}{rgb}{1,1,1}

\lstdefinestyle{mystyle}{
    backgroundcolor=\color{backcolour},   
    commentstyle=\color{codegreen},
    keywordstyle=\color{magenta},
    numberstyle=\tiny\color{codegray},
    stringstyle=\color{codepurple},
    basicstyle=\ttfamily\footnotesize,
    breakatwhitespace=false,         
    breaklines=true,                 
    captionpos=b,                    
    keepspaces=true,                 
    numbers=left,                    
    numbersep=5pt,                  
    showspaces=false,                
    showstringspaces=false,
    showtabs=false,                  
    tabsize=2
}

\lstset{style=mystyle}

\begin{document}

\begin{frame}{Page 2.}
    \begin{figure}[h]
        \centering
        \subfloat[Georges Matheron.]{\includegraphics[scale=0.6]{//home/etienne/Bureau/TIPE/Final/Images/Georges_Matheron.jpg}}
        \quad
        \subfloat[École des Mines de Paris.]{\includegraphics[scale=0.5]{//home/etienne/Bureau/TIPE/Final/Images/Mines_Paris.jpg}}
    \end{figure}
\end{frame}

\begin{frame}{Page 8.}
    \begin{figure}[h]
        \centering
        \subfloat[Image d'origine.]{\includegraphics[scale=0.32]{//home/etienne/Bureau/TIPE/Final/Images/singe.png}}
        \quad
        \subfloat[Dilatation par un disque de rayon 3.]{\includegraphics[scale=0.32]{//home/etienne/Bureau/TIPE/Final/Images/singe_dilate_3.png}}
    \end{figure}
\end{frame}

\begin{frame}{Page 9.}
    \begin{figure}[h]
        \centering
        \subfloat[Image d'origine.]{\includegraphics[scale=0.32]{//home/etienne/Bureau/TIPE/Final/Images/singe.png}}
        \quad
        \subfloat[Érosion par un disque de rayon 3.]{\includegraphics[scale=0.32]{//home/etienne/Bureau/TIPE/Final/Images/singe_erode_3.png}}
    \end{figure}
\end{frame}

\begin{frame}{Page 10.}
    \begin{figure}[h]
        \centering
        \subfloat[Image d'origine.]{\includegraphics[scale=0.32]{//home/etienne/Bureau/TIPE/Final/Images/singe.png}}
        \quad
        \subfloat[Ouverture par un disque de rayon 3.]{\includegraphics[scale=0.32]{//home/etienne/Bureau/TIPE/Final/Images/singe_ouvert_3.png}}
    \end{figure}
\end{frame}

\begin{frame}{Page 11.}
    \begin{figure}[h]
        \centering
        \subfloat[Image d'origine.]{\includegraphics[scale=0.32]{//home/etienne/Bureau/TIPE/Final/Images/singe.png}}
        \quad
        \subfloat[Fermeture par un disque de rayon 3.]{\includegraphics[scale=0.32]{//home/etienne/Bureau/TIPE/Final/Images/singe_ferme_3.png}}
    \end{figure}
\end{frame}

\begin{frame}{Page 17.}
    \begin{figure}[h]
        \centering
        \subfloat[$\alpha = 0,33; \beta = 0,33; \mathcal E$ et $\mathcal D$.]{\includegraphics[scale=0.32]{//home/etienne/Bureau/TIPE/Final/Images/singe_rehausse_equi_ero_dil.png}}
        \quad
        \subfloat[$\alpha = 0,33; \beta = 0,33; f_{\mathcal B}$ et $f^{\mathcal B}$.]{\includegraphics[scale=0.32]{//home/etienne/Bureau/TIPE/Final/Images/singe_rehausse_equi_ouv_fer.png}}
    \end{figure}
\end{frame}

\begin{frame}{Page 18.}
    \begin{figure}[h]
        \centering
        \subfloat[Image d'origine.]{\includegraphics[scale=0.32]{//home/etienne/Bureau/TIPE/Final/Images/singe.png}}
        \quad
        \subfloat[Gradient morphologique.]{\includegraphics[scale=0.32]{//home/etienne/Bureau/TIPE/Final/Images/singe_gradient.png}}
    \end{figure}
\end{frame}

\begin{frame}{Page 19.}
    \begin{figure}[h]
        \centering
        \subfloat[Image d'origine.]{\includegraphics[scale=0.32]{//home/etienne/Bureau/TIPE/Final/Images/singe.png}}
        \quad
        \subfloat[Chapeau haut-de-forme par un disque de rayon 3.]{\includegraphics[scale=0.32]{//home/etienne/Bureau/TIPE/Final/Images/singe_chapeau_hdf.png}}
    \end{figure}
\end{frame}

\begin{frame}{Page 20.}
    \begin{center}
    \includegraphics[scale=0.25]{//home/etienne/Bureau/TIPE/Final/Images/rouge_femme.png}
    \includegraphics[scale=0.25]{//home/etienne/Bureau/TIPE/Final/Images/vert_femme.png}
    \includegraphics[scale=0.25]{//home/etienne/Bureau/TIPE/Final/Images/bleu_femme.png}
    \includegraphics[scale=0.25]{//home/etienne/Bureau/TIPE/Final/Images/couleur_femme.png}\\        
    \end{center}
    Décomposition des images en son spectre de couleurs.
\end{frame}

\begin{frame}{Page 23.}
    \begin{figure}[h]
        \centering
        \subfloat[Érosion par un carré de côté 3.]{\includegraphics[scale=0.4]{//home/etienne/Bureau/TIPE/Final/Images/singe_erode_3_carre.png}}
    \end{figure}
\end{frame}

\end{document}