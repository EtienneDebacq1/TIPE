\documentclass[12pt]{article}
\usepackage{graphicx} % Required for inserting images
\usepackage{amsfonts}
\usepackage{ragged2e}
\usepackage[left=2cm, right=2cm, top=2cm, bottom=2cm]{geometry}
\usepackage{color}
\usepackage{amsmath, amssymb}
\usepackage{lastpage}
\usepackage{fancyhdr}
\usepackage[T1]{fontenc}
\usepackage{hyperref}
\usepackage{stmaryrd}
\usepackage{tikz}

\usepackage{algorithm}
\usepackage{algpseudocode}

\usepackage{listings}

\definecolor{codegreen}{rgb}{0,0.6,0}
\definecolor{codegray}{rgb}{0.5,0.5,0.5}
\definecolor{codepurple}{rgb}{0.58,0,0.82}
\definecolor{backcolour}{rgb}{0.95,0.95,0.92}

\lstdefinestyle{mystyle}{
    backgroundcolor=\color{backcolour},   
    commentstyle=\color{codegreen},
    keywordstyle=\color{magenta},
    numberstyle=\tiny\color{codegray},
    stringstyle=\color{codepurple},
    basicstyle=\ttfamily\footnotesize,
    breakatwhitespace=false,         
    breaklines=true,                 
    captionpos=b,                    
    keepspaces=true,                 
    numbers=left,                    
    numbersep=5pt,                  
    showspaces=false,                
    showstringspaces=false,
    showtabs=false,                  
    tabsize=2
}

\lstset{style=mystyle}

\def\checkmark{\tikz\fill[scale=0.4](0,.35) -- (.25,0) -- (1,.7) -- (.25,.15) -- cycle;}

\title{Texte pour la présentation.}
\author{Étienne Debacq}
\date{6 novembre 2024 - 10 juin 2025.}

\setlength{\headheight}{15pt}
\pagestyle{fancy}
\cfoot{\thepage\ sur \pageref*{LastPage}}


\hypersetup{
    colorlinks=true,
    citecolor=black,
    linktoc=all,
    linkcolor=black
}

\begin{document}
\maketitle

\thispagestyle{fancy}
\vspace{20pt}

\fancyhead[L]{Étienne Debacq}
\fancyhead[R]{MPI* Paul Valéry}
\fancyhead[C]{TIPE}

%====================================================================

\section{Présentation de la théorie.}

\subsection{Introduction des quatre opérations fondamentales}

\quad \underline{Diapo 1 :} Bonjour, je vais vous présenter mon TIPE sur le thème << Transition, transformation, conversion >> sur la morphologie mathématique.\\

\underline{Diapo 2 :} La morphologie mathématique est une branche très récente des mathématiques appliquées car elle a été théorisée dans la seconde moitié du $\text{XX}^{\text{ème}}$ mais est pour autant très fructueuse grâce à sa facilité d'implémentation et les résultats fournis. \\

\underline{Diapo 3 :} Mon TIPE est animé par la question suivante : \\
$\longrightarrow$ Dans quelle mesure la morphologie mathématique permet-elle d'obtenir des transformations efficaces pour les images grayscale ? \\

\underline{Diapo 4 :} Pour y répondre, j'ai trouvé trois objectifs à réaliser : \\
$\longrightarrow$ Une étude de la morphologie mathématique. \\
$\longrightarrow$ Une implémentation des outils étudiés dans le langage C. \\
$\longrightarrow$ Une recherche d'algorithmes optimisés. \\

\underline{Diapo 5 :} J'ai divisé le travail en trois parties afin de rendre l'étude la plus digeste possible. \\

\underline{Diapo 6 :} La morphologie mathématique s'applique à tout types d'objets : qu'ils soient discrets, donc un point de vue ensembliste, ou continus, donc un point de vue fonctionnel. Le fait qu'il y ait deux points de vue implique des différences dans l'étude des images binaires, donc en noir et blanc, et des images grayscale, donc en teintes de gris. Ce sont ces dernières qui m'ont intéressées pour mon étude. \\
Cependant, il est possible dans le cas du traitement d'image, de restreindre l'étude à des sous-ensembles de $n$-uplets d'entiers naturels au lieu de $n$-uplets de nombres réels, mais aussi de se contenter, dans notre cas, d'une spécification pour $n=2$. \\

\underline{Diapo 7 :} On appelle élément structurant une sous-partie de $\mathbb N^2$ dont la forme ainsi que la taille sont fixées. Aussi, on lui assigne un point de l'espace arbitraire qu'on appelle centre ou origine de $\mathcal B$ et dont le rôle est de gérer la manipulation de cet ensemble. \\
Cet élement structurant nous permet de modifier l'image à notre guise, comme pourrait l'être la mine d'un crayon. \\
On peut manipuler de nombreuses formes comme des disques, des carrés, ou toutes autres formes paramétrables. La forme la plus utilisée et la plus simplement paramétrable est le disque, dont voici un exemple, et dont l'origine est son propre centre. \\

\underline{Diapo 8 :} Une des quatre opérations élémentaire est la dilatation d'une fonction $f$ par un élément structurant $\mathcal B$ notée $\mathcal D(f, \mathcal B)$ et dont voici la définition reposant sur la notion de sup et de translation.\\
Cette opération a un effet de dilatation, c'est à dire qu'elle propage les maxima locaux des valeurs utilisées et donc augmente les niveaux de gris, ce qui provoque un effet de flou clair. \\

\underline{Diapo 9 :} L'érosion d'une fonction $f$ par un élément structurant $\mathcal B$ est semblable par sa définition à la dilatation fonctionnelle à laquelle le sup est remplacé par un inf tout en gardant les translations. \\
Cette opération a un effet opposé à la dilatation, c'est à dire qu'elle propage les minima locaux des valeurs utilisées et donc diminue les niveaux de gris, ce qui provoque un effet de flou sombre. \\

\underline{Diapo 10 :} Une autre opération fondamentale est l'ouverture de $f$ par $\mathcal B$. Celle-ci combine les deux précédentes afin d'en créer une nouvelle. \\
Cette opération a un effet plus subtil que les précédentes car elle permet d'éliminer les composantes plus petites que l'élément structurant et donc d'enlever du bruit par exemple. \\

\underline{Diapo 11 :} La dernière opération fondamentale est la fermeture de $f$ par un élément structurant $\mathcal B$ qui est définie en inversant la dilatation et l'érosion dans le sens de lecture.\\
Cette opération a un effet similaire à l'ouverture en remplissant cette fois les petits trous dans l'image afin de la rendre dans un certain sens convexe.\\

\underline{Diapo 12 :} Je me suis limité dans mon étude à l'utilisation de disques car ceux-ci proposent des résultats plus flagrants et sont plus facilement paramétrables. \\ 
Cette restriction permet aussi une simplification majeure qu'est le fait que $\mathcal S_{\mathcal B} = \mathcal B$ dans le cas où l'origine est le centre du disque. Ceci permet alors d'éviter d'avoir à calculer sans cesse le symétrique de $\mathcal B$ par rapport à son centre et donc permet d'alléger les calculs.

\subsection{Choix du langage}

\underline{Diapo 13 :} Le choix du langage C s'est fait très naturellement pour plusieurs raisons propres à mon étude. \\ 
Ce langage est en effet plus rapide que d'autres tel Python lors de l'exécution, ce qui permet dans mon cas d'obtenir rapidement un résultat. \\
De plus, celui-ci permet un contrôle total de la mémoire avec les fonctions malloc et free, pour respectivement allouer et libérer la mémoire, contrôle inexistant dans les autres langages de programmations que je connais.\\
Enfin, ce langage étant au programme de ma filièrme, cela m'a motivé à l'utiliser afin de m'améliorer tout au long de l'année.

%====================================================================

\section{Implémentation des opérations fondamentales.}

\subsection{Définition des types}

\quad \underline{Diapo 14 :} Mon utilisation du langage C m'a permis de faire appel aux types structurés et donc de pouvoir librement définir des nouveaux types qui me seront utiles. \\
Ainsi, un type Image qui contient les caractéristiques de l'image, c'est à dire ses dimensions ainsi que son contenu me permet de créer les images figurant dans ma présentation. \\
Les types Fonction et Structurant sont eux aussi des matrices de $\mathcal M_{L_0, L_1}([\![0;255]\!])$ et des matrices de $\mathcal M_{2r}(\mathbb Z/2\mathbb Z)$ permettent de manipuler les opérations précédemment définies afin de transformer les images. Structurant est aussi munie d'un pointeur sur entier qui est l'origine.

\subsection{Complexité}

\quad \underline{Diapo 15 :} Pour l'étude de la complexité, il est important de remarquer que $r$ est très petit devant les autres dimensions afin d'avoir des images encore lisibles tout en minimisant le coût des opérations.\\
Ceci entraîne alors la négligeabilité totale des opérations sur les éléments structurants, donc en $\mathcal O(1)$. \\
Le résultat important sur la complexité des opérations en morphologique mathématique est que celles-ci sont sous cette hypothèse toutes en $\mathcal O(L_0\times L_1)$. 

%====================================================================

\section{Création d'opérations complexes.}

\quad \underline{Diapo 16 :} Une nouvelle opération qu'on peut définir est le rehaussement de contraste d'une fonction $f$ par la définition suivante. L'hypothèse $\alpha + \beta < 1$ est nécessaire pour que le cas du milieu puisse exister. \\
Cette opération n'utilise par définition aucune des quatre opérations foncdamentales présentées plus haut. Néanmoins, la morphologie mathématique fournit des fonctions $\underline{f}$ et $\overline{f}$ nécessaire dans la définition de $\mathcal C_{f}$. \\
On peut en effet avoir recours à l'ouverture ou l'érosion en fonction minorante et la fermeture ou la dilatation en fonction majorante justement car ces fonctions sont extensives ou anti-extensives, c'est à dire croissantes ou décroissantes ici.\\
Cette opération à pour effet de basculer les valeurs de l'image selon celles de trois cas possibles, et donc de renforcer le contraste de l'image. \\

\underline{Diapo 17 :} On fixe $\alpha = 0,33$ et $\beta = 0,33$. Ces nombres respectent la condition et on choisit ici l'ouverture et la fermeture selon un disque de rayon 3. Dans cette configuration, l'effet est assez minime car les deux fonctions utilisées ont des répercutions assez subtiles sur l'image. \\
Cette fois-ci, avec les mêmes valeurs pour $\alpha$ et $\beta$, j'ai choisi d'utiliser l'autre couple d'opérations fondamentales ce qui donne cette fois un résultat très visible. En effet, ces opérations étant bien moins subtiles, le résultat ne l'est pas non plus. \\
On peut alors tester d'autres couples de fonctions ou de valeurs tant qu'ils respectent les hypothèses pour obtenir d'autres résultats. \\ 

\underline{Diapo 18 :} Une autre opération très utile est le gradient morphologique. \\
Contrairement aux autres, celle-ci ne modifie pas la forme de l'image en l'agrandissant ou en la rapetissant. En effet, cette opération va permettre de repérer de manière marquée les plus grosses variations dans les valeurs de l'image. Ainsi, dans l'exemple utilisé, on peut voir que les teintes de gris varient très fortement sur les contours du chapeau et très peu au niveau des épaules. \\

\underline{Diapo 19 :} Enfin, la dernière opération que j'ai décidé d'implémenter est la transformation dite du chapeau haut-de-forme, qui est définie comme la différence entre $f$ et son ouverture par $\mathcal B$. \\
Cette opération, à la manière du gradient morphologique, ne fait qu'extraire des composantes de l'image. Ici, les composantes relevées sont les pics plus petits que l'élément structurant qui se situent sur les plumes sur l'image et l'opération sélectionne les contours abrupts.

%====================================================================

\section{Conclusion.}

\subsection{Adaptation aux couleurs.}

\quad \underline{Diapo 20 :} Ainsi, une fois les images grayscale traitées, il est possible, par une approche naïve, de traiter les images couleurs comme la superposition de ses trois composantes primaires auxquelles on applique la théorie vue précédemment. Cette approche a l'inconvénient de ne pas respecter l'identité visuelle de l'image, pouvant causer des artefacts ou des changements de couleurs.

\subsection{Quelques applications.}

\quad \underline{Diapo 21 :} Finalement, la morphologie mathématique a de nombreuses ouvertures vers d'autres domaines importants des sciences et des technologies utiles au quotidien telles la reconnaissance de formes avec la notion de squelette que je n'ai ici pas développée, le tamisage des sols ou encore les lignes de partage des eaux. Ces deux dernières applications ayant motivé le développement de la morphologie mathématique.

\end{document}
